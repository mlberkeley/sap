\documentclass[11pt]{exam}
\noprintanswers
\usepackage{array,multirow,graphicx}
\graphicspath{ {images/} }
\usepackage{amsmath,amssymb,amsthm,bm,hyperref, enumitem}
\usepackage[parfill]{parskip}
\usepackage[margin=1in]{geometry}
\usepackage{color}
\usepackage{pdfpages}
\newtheoremstyle{quest}{\topsep}{\topsep}{}{}{\bfseries}{}{ }{\thmname{#1}\thmnote{ #3}.}
\theoremstyle{quest}
\newtheorem*{definition}{Definition}
\newtheorem*{theorem}{Theorem}
\newtheorem*{question}{Problem}
\newtheorem*{exercise}{Exercise}
\newtheorem*{challengeproblem}{Challenge Problem}

\newcommand{\name}{ML@B}

% \title{
% \Large \name
% \\\vspace{10pt}
% \\\vspace{10pt}
% \duedate
% \author{Phillip Kuznetsov}
% }


\markright{\name\hfill Homework \hw\hfill}

%% If you want to define a new command, you can do it like this:
\newcommand{\Q}{\mathbb{Q}}
\newcommand{\R}{\mathbb{R}}
\newcommand{\Z}{\mathbb{Z}}
\newcommand{\C}{\mathbb{C}}
\newcommand{\norm}[2]{\|#1\|_#2}
\newcommand{\deriv}[2]{\frac{\partial #1}{\partial #2}}
\newcommand{\sign}[1]{\text{sign}(#1)}
\newcommand{\supp}[2]{#1^{(#2)}}
\newcommand{\ddelta}[2]{\supp{#1}{#2}}
%% If you want to use a function like ''sin'' or ''cos'', you can do it like this
%% (we probably won't have much use for this)
% \DeclareMathOperator{\sin}{sin}   %% just an example (it's already defined)

\begin{document}
\begin{figure}[t]
    \centering
    \includegraphics[width=5cm]{brain}\\
    \Large ML@B\\
    SAP Algorithm Outline
\end{figure}
% \maketitle
\vspace{}
Notation:
\begin{itemize}
    \item $D$: Data matrix \\
    \begin{itemize}
        \item $D_i$: Datapoint with format $\{\text{close price} - \text{open price}, \text{low}, \text{high}, \text{volume},F_{i-1}\}$
    \end{itemize}
    \item $F$: Trade value% TODO add possible values \\
    \begin{itemize}
        \item $F_i = D \cdot \Theta = D_{i0}\Theta_0 + D_{i1}\Theta_1 +D_{i2}\Theta_2 +D_{i3}\Theta_3 +D_{i4}\Theta_4$
        \item The 0th - 3rd components are calculated at initialization, the fourth component is calculated later
        \item $F_i^\prime =  D_{i0}\Theta_0 + D_{i1}\Theta_1 +D_{i2}\Theta_2 +D_{i3}\Theta_3$
        \item $F_i = F_i^\prime + \Theta_4 F_{i-1} = \sum_{j=0}^i \Theta_4^{i-j} F_i^\prime$

    \end{itemize}
    \item $R_i$: The reward function at time step $i$
    \begin{itemize}
        \item $R_i = F_i D_0 - \partial(F_i - F_{i-1})$
    \end{itemize}
    \item Gradient Equations
    \begin{itemize}
        \item $$\deriv{R_t}{F_t} = \left\{\begin{matrix}
-1 \text{ if } F_{t-1} < F_t \\
+1 \text{ else}
\end{matrix}\right$$
        \item $$\deriv{R_t}{F_{t-1}} =D_{t0}-  \left\{\begin{matrix}
-1 \text{ if } F_{t-1} < F_t \\
+1 \text{ else}
\end{matrix}\right = D_{t0} - \deriv{R_t}{F_t}$$

        \item $$\deriv{F_{t}}{\Theta} = D_t$$
        \item $$\deriv{U_{t}}{\Theta} = \sum_{i=1}^t \deriv{U_t}{R_t}\left(\deriv{R_t}{F_t}\deriv{F_{t}}{\Theta} + \deriv{R_t}{F_{t-1}}\deriv{F_{t-1}}{\Theta} \right)
    \end{itemize}
    \item $U$: % TODO add u stuff

\end{itemize}
\end{document}
