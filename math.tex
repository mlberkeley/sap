\documentclass[11pt]{exam}
\noprintanswers
\usepackage{array,multirow,graphicx}
\graphicspath{ {images/} }
\usepackage{amsmath,amssymb,amsthm,bm,hyperref, enumitem}
\usepackage[parfill]{parskip}
\usepackage[margin=1in]{geometry}
\usepackage{color}
\usepackage{pdfpages}
\newtheoremstyle{quest}{\topsep}{\topsep}{}{}{\bfseries}{}{ }{\thmname{#1}\thmnote{ #3}.}
\theoremstyle{quest}
\newtheorem*{definition}{Definition}
\newtheorem*{theorem}{Theorem}
\newtheorem*{question}{Problem}
\newtheorem*{exercise}{Exercise}
\newtheorem*{challengeproblem}{Challenge Problem}

\newcommand{\name}{ML@B}

% \title{
% \Large \name
% \\\vspace{10pt}
% \\\vspace{10pt}
% \duedate
% \author{Phillip Kuznetsov}
% }


\markright{\name\hfill Homework \hw\hfill}

%% If you want to define a new command, you can do it like this:
\newcommand{\Q}{\mathbb{Q}}
\newcommand{\R}{\mathbb{R}}
\newcommand{\Z}{\mathbb{Z}}
\newcommand{\C}{\mathbb{C}}
\newcommand{\norm}[2]{\|#1\|_#2}
\newcommand{\deriv}[2]{\frac{\partial #1}{\partial #2}}
\newcommand{\sign}[1]{\text{sign}(#1)}
\newcommand{\supp}[2]{#1^{(#2)}}
\newcommand{\ddelta}[2]{\supp{#1}{#2}}
%% If you want to use a function like ''sin'' or ''cos'', you can do it like this
%% (we probably won't have much use for this)
% \DeclareMathOperator{\sin}{sin}   %% just an example (it's already defined)

\begin{document}
\begin{figure}[t]
    \centering
    \includegraphics[width=5cm]{brain}\\
    \Large ML@B\\
    SAP Algorithm Outline
\end{figure}
% \maketitle
\vspace{}
Notation:
\begin{itemize}
    \item $D$: Data matrix \\
    \begin{itemize}
        \item $D_i$: Datapoint of format $\{\text{close price} - \text{open price}, \text{low}, \text{high}, \text{volume},F_{i-1}\}$
    \end{itemize}
    \item $F$: Trade value% TODO add possible values \\
    \begin{itemize}
        \item $F_i = D_i \cdot \Theta = D_{i_0}\Theta_0 + D_{i_1}\Theta_1 +D_{i_2}\Theta_2 +D_{i_3}\Theta_3 +D_{i_4}\Theta_4$
        \item The 0th - 3rd components are calculated at initialization, the fourth component is calculated later. $D_{i_4}$ is $F_{i-1}$ which is currently not fully calculated.
        \item $F_i^\prime =  D_{i_0}\Theta_0 + D_{i_1}\Theta_1 +D_{i_2}\Theta_2 +D_{i_3}\Theta_3$
        \item $F_i = F_i^\prime + \Theta_4 F_{i-1} = \sum_{j=0}^i \Theta_4^{i-j} F_j^\prime$ (This form is very similar to the Prefix Sum)

    \end{itemize}
    \item $R_i$: The reward function at time step $i$
    \begin{itemize}
        \item $R_i = F_{i-1} D_0 - \delta|F_i - F_{i-1}|$
    \end{itemize}
    \item $S^\prime$: The Differential Sharpe ratio % TODO add u stuff
    \begin{itemize}
        \item $S$ the Sharp ratio is obtained by considering $S_T=\frac{\text{Average}(R_t)}{\text{Standard Deviation}(R_t)}$
        \item $S^\prime $ is obtained by considering exponential moving averages of the returns and standard deviation of returns above, and expanding to first order in the decay rate $\eta$:
        \item $$\deriv{S_{t}}{\eta} = \frac{B_{t-1}\Delta A_t - \frac{1}{2}A_{t-1}\Delta B_t}{(B_{t-1}-A_{t-1}^2)^{3/2}}$$
        \item $A_t = A_{t-1} + \eta \Delta A_t = A_{t-1} + \eta (R_t - A_{t-1})$
        \item $B_t = B_{t-1} + \eta \Delta B_t = B_{t-1} + \eta (R_t^2 - B_{t-1})$
    \end{itemize}
    \item Performance function
    \begin{itemize}
        \item Want to maximize performance functions instead of minimizing a cost function
        \item $U_T = U(R_1,R_2,\cdot,R_t)$ is the cumulative result of the differential sharpe ratio
        \item For on-line learning we'll only worry about $\deriv{U_t}{R_t}$ which is the differential sharpe ratio for only the current timestep
    \end{itemize}
    \item Gradient Equations Derived from Above
    \begin{itemize}
        \item $$\deriv{R_t}{F_t} = \left\{\begin{matrix}
-1 \text{ if } F_{t-1} < F_t \\
+1 \text{ else}
\end{matrix}\right$$
        \item $$\deriv{R_t}{F_{t-1}} =D_{t_0}-  \left\{\begin{matrix}
-1 \text{ if } F_{t-1} < F_t \\
+1 \text{ else}
\end{matrix}\right = D_{t_0} - \deriv{R_t}{F_t}$$

        \item $$\deriv{F_{t}}{\Theta} = D_t$$
        \item $$\deriv{U_t}{R_t}=\frac{B_{t-1}-A_{t-1} R_t}{(B_{t-1} - A_{t-1}^2)^{3/2}}$$
        \item $$\deriv{U_{t}}{\Theta} = \sum_{i=1}^t \deriv{U_t}{R_t}\left(\deriv{R_t}{F_t}\deriv{F_{t}}{\Theta} + \deriv{R_t}{F_{t-1}}\deriv{F_{t-1}}{\Theta} \right)$$
    \end{itemize}
    
    

\end{itemize}
\end{document}
